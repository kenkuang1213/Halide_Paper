\chapter{Background}

\section{Halide}
    Halide is a new programming language designed to make it easier to write high-performance image processing code on modern machines. Its current front end is embedded in C++. Compiler targets include x86/SSE, ARM v7/NEON, CUDA, Native Client, and OpenCL.

    Halide represents a systematic model of the tradeoff space fundamental to stencil pipelines, a schedule representation which describes concrete points in this space for each stage in an image processing pipeline, and an optimizing compiler for the Halide image processing language that synthesizes high performance implementations from a Halide algorithm and a schedule. Combining this compiler with stochastic search over the space of schedules enables terse, composable programs to achieve state-of-the-art performance on a wide range of real image processing pipelines, and across different hardware architectures, including multicores with SIMD, and heterogeneous CPU and GPU execution.

\section{OpenCL}
    OpenCL (Open Computing Language) is an open royalty-free standard for general purpose parallel programming across CPUs, GPUs and other processors, giving software developers portable and efficient access to the power of these heterogeneous processing platforms.

    OpenCL supports a wide range of applications, ranging from embedded and consumer software to HPC solutions, through a low-level, high-performance, portable abstraction. By creating an efficient, close-to-the-metal programming interface, OpenCL will form the foundation layer of a parallel computing ecosystem of platform-independent tools, middleware and applications. OpenCL is particularly suited to play an increasingly significant role in emerging interactive graphics applications that combine general parallel compute algorithms with graphics rendering pipelines.
    
    OpenCL consists of an API for coordinating parallel computation across heterogeneous processors; and a cross-platform programming language with a well- specified computation environment. The OpenCL standard: Supports both data- and task-based parallel programming models, Utilizes a subset of ISO C99 with extensions for parallelism, Defines consistent numerical requirements based on IEEE 754, Defines a configuration profile for handheld and embedded devices, Efficiently interoperates with OpenGL, OpenGL ES and other graphics APIs.
